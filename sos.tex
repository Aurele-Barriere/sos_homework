\documentclass{article}

\usepackage{proof}
\usepackage{amsmath,amssymb,amsthm,textcomp}
\usepackage{fullpage}

\newcommand{\var}{\mathbf{Var}}
\newcommand{\state}{\mathbf{State}}
\newcommand{\comm}{\mathbf{Comm}}
\newcommand{\A}{\mathcal{A}}
\newcommand{\Aa}[2]{\mathcal{A}\left[#1\right]\left[#2\right]}
\newcommand{\N}{\mathcal{N}}
\newcommand{\true}{\mathbf{tt}}
\newcommand{\false}{\mathbf{ff}}
\newcommand{\expr}{\mathbf{exp}}
\newcommand{\Rule}[3]{\infer[\textsc{#1}]{#2}{#3}}

\newcommand{\ie}{\emph{i.e.}}


\title{Homework SOS}
\author{Aur\`ele Barri\`ere \& Florestan De Moor}
\date{November 4th 2018}

\begin{document}

\maketitle

\section*{Question 1}

%\textsc{Rule 1}\quad\infer{A}{B & C}

We define the $\state{}$ domain as follows:

\[
\state{} = \left(\var{}  \longrightarrow \mathbb{Z} \right) \times \left( \mathbb{N} \longrightarrow \mathbb{Z} \right)
\]

\ie{} each variable is assigned to an integer value, and each address of the memory (which is a natural) is mapped to an integer.

We then define our set of configurations:

\[
\Gamma = \lbrace \left( c, (\sigma, \delta) \right) \mid c \in \comm{}, (\sigma, \delta) \in \state{} \rbrace \cup \Gamma_{final}
\]

where $\Gamma_{final} = \state{}$.

We define the denotational semantics of expressions $\A{}: \expr{} \longrightarrow \state{} \longrightarrow \mathbb{Z} \cup \mathbb{B}$ recursively as follows:

\begin{align*}
	\Aa{x}{\sigma, \delta} &= \sigma(x) \\
	\Aa{n}{\sigma, \delta} &= \N(n) \\
	\Aa{e_1 \circ e_2}{\sigma, \delta} &= \Aa{e_1}{\sigma, \delta} \bullet \Aa{e_2}{\sigma, \delta} \\
	\Aa{\texttt{true}}{\sigma, \delta} &= \true{} \\
	\Aa{\texttt{false}}{\sigma, \delta} &= \false{} \\
	\Aa{\texttt{not}~e}{\sigma, \delta} &= \neg \Aa{e}{\sigma, \delta} \\
	\Aa{e1~\texttt{and}~e_2}{\sigma, \delta} &= \Aa{e_1}{\sigma, \delta} \wedge \Aa{e_2}{\sigma, \delta} \\
	\Aa{e1~\texttt{or}~e_2}{\sigma, \delta} &= \Aa{e_1}{\sigma, \delta} \vee \Aa{e_2}{\sigma, \delta} \\
	\Aa{[e]}{\sigma, \delta} &= \delta(\sigma(e)) \\
\end{align*}

where $\bullet \in \lbrace +, -, \times, =, \leq \rbrace$ is the semenatic operator associated to the syntaxic operator $\circ$.

We can now define a structural operational semantics of commands:

\[
	\Rule{Skip}{\texttt{skip}, \sigma, \delta \longrightarrow \sigma, \delta}{} \hspace*{1cm}
	\Rule{Seq1}{c_1 ; c_2, \sigma, \delta \longrightarrow c_2, \sigma', \delta'}{c_1, \sigma, \delta \longrightarrow \sigma', \delta'} \hspace*{1cm}
	\Rule{Seq2}{c_1 ; c_2, \sigma, \delta \longrightarrow c_1 ; c_2, \sigma', \delta'}{c_1, \sigma, \delta \longrightarrow c_1', \sigma', \delta'}
\]
\[
	\Rule{AssignVar}{x := e, \sigma, \delta \longrightarrow \sigma[x \mapsto \Aa{e}{\sigma, \delta}], \delta}{x \in \var{}} \hspace*{1cm}
	\Rule{AssignMem}{[e_1] := e_2, \sigma, \delta \longrightarrow \sigma, \delta[\Aa{e_1}{\sigma, \delta} \mapsto \Aa{e_2}{\sigma, \delta}]}{}
\]
\[
	\Rule{If1}{\texttt{if}~e~\texttt{then}~c_1~\texttt{else}~c_2, \sigma, \delta \longrightarrow c_1, \sigma, \delta}{} \quad \text{if $\Aa{e}{\sigma, \delta} = \true{}$}
\]
\[
	\Rule{If2}{\texttt{if}~e~\texttt{then}~c_1~\texttt{else}~c_2, \sigma, \delta \longrightarrow c_2, \sigma, \delta}{} \quad \text{if $\Aa{e}{\sigma, \delta} = \false{}$}
\]
\[
	\Rule{While}{\texttt{while}~e~\texttt{do}~c, \sigma, \delta \longrightarrow \texttt{if}~e~\texttt{then}~(c; \texttt{while}~e~\texttt{do}~c)~\texttt{else}~\texttt{skip}, \sigma, \delta}{} 
\]

\section*{Question 2}
	
\section*{Question 3}


\end{document}
