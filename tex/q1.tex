We define the $\state{}$ domain as follows:

\[
\state{} = \left(\var{}  \longrightarrow \mathbb{Z} \cup \mathbb{B} \right) \times \left( \mathbb{N} \longrightarrow \mathbb{Z} \right)
\]

\ie{} each variable is assigned to an integer or boolean  value, and each address of the memory (which is a natural) is mapped to an integer.

We then define our set of configurations:

\[
\C = \lbrace \left( c, (\sigma, \delta) \right) \mid c \in \comm{}, (\sigma, \delta) \in \state{} \rbrace \cup \C_{final}
\]

where $\C_{final} = \state{}$.

We define the denotational semantics of expressions $\A{}: \expr{} \longrightarrow \state{} \longrightarrow \mathbb{Z} \cup \mathbb{B}$ recursively as follows:

\begin{align*}
	\Aa{x}{\sigma, \delta} &= \sigma(x) \\
	\Aa{n}{\sigma, \delta} &= \N(n) \\
	\Aa{e_1 \circ e_2}{\sigma, \delta} &= \Aa{e_1}{\sigma, \delta} \bullet \Aa{e_2}{\sigma, \delta} \\
	\Aa{\texttt{true}}{\sigma, \delta} &= \true{} \\
	\Aa{\texttt{false}}{\sigma, \delta} &= \false{} \\
	\Aa{\texttt{not}~e}{\sigma, \delta} &= \neg \Aa{e}{\sigma, \delta} \\
	\Aa{e1~\texttt{and}~e_2}{\sigma, \delta} &= \Aa{e_1}{\sigma, \delta} \wedge \Aa{e_2}{\sigma, \delta} \\
	\Aa{e1~\texttt{or}~e_2}{\sigma, \delta} &= \Aa{e_1}{\sigma, \delta} \vee \Aa{e_2}{\sigma, \delta} \\
	\Aa{[e]}{\sigma, \delta} &= \delta(\sigma(e)) \\
\end{align*}

where $\bullet \in \lbrace +, -, \times, =, \leq \rbrace$ is the semantic operator associated to the syntactic operator $\circ$. $\A{}$ is a partial function. For instance, $\vee$ is defined as usual, only on boolean values.

We can now define a structural operational semantics of commands:

\[
	\Rule{Skip}{\texttt{skip}, \sigma, \delta \longrightarrow \sigma, \delta}{} \hspace*{1cm}
	\Rule{Seq1}{c_1 ; c_2, \sigma, \delta \longrightarrow c_2, \sigma', \delta'}{c_1, \sigma, \delta \longrightarrow \sigma', \delta'} \hspace*{1cm}
	\Rule{Seq2}{c_1 ; c_2, \sigma, \delta \longrightarrow c_1 ; c_2, \sigma', \delta'}{c_1, \sigma, \delta \longrightarrow c_1', \sigma', \delta'}
\]
\\
\[
	\Rule{AssignVar}{x := e, \sigma, \delta \longrightarrow \sigma[x \mapsto \Aa{e}{\sigma, \delta}], \delta}{x \in \var{}}
\]
\\
\[
	\Rule{AssignMem}{[e_1] := e_2, \sigma, \delta \longrightarrow \sigma, \delta[\Aa{e_1}{\sigma, \delta} \mapsto \Aa{e_2}{\sigma, \delta}]}{}
\]
\\
\[
	\Rule{If1}{\texttt{if}~e~\texttt{then}~c_1~\texttt{else}~c_2, \sigma, \delta \longrightarrow c_1, \sigma, \delta}{} \quad \text{if $\Aa{e}{\sigma, \delta} = \true{}$}
\]
\\
\[
	\Rule{If2}{\texttt{if}~e~\texttt{then}~c_1~\texttt{else}~c_2, \sigma, \delta \longrightarrow c_2, \sigma, \delta}{} \quad \text{if $\Aa{e}{\sigma, \delta} = \false{}$}
\]
\\
\[
	\Rule{While}{\texttt{while}~e~\texttt{do}~c, \sigma, \delta \longrightarrow \texttt{if}~e~\texttt{then}~(c; \texttt{while}~e~\texttt{do}~c)~\texttt{else}~\texttt{skip}, \sigma, \delta}{} 
\]

\paragraph{Comments}
We define our $\state$ domain as a cartesian product for readability, the second function only describing the memory.
Final configurations are simply states, with no more commands to execute.

As expressions define both booleans and integer expressions, our $\A{}$ function can be evaluated in $\mathbb{Z}$ or $\mathbb{B}$.

Our semantics is defined such that some programs may not evaluate to a final configuration. We give below a few examples:

\texttt{[-3] := 0} will not evaluate, because for any state $(\sigma,\delta)$, $\delta$ is only defined on $\mathbb{N}$. Thus the \textsc{AssignMem} rule can't be applied.

\texttt{x := true + 2} won't make any progress, because $\Aa{true + 2}{\sigma,\delta}$ isn't defined for any $(\delta,\sigma)$.

\texttt{if 1 then c\_1 else c\_2} won't make any progress, because $\Aa{1}{\sigma,\delta}$ is neither equal to \textbf{tt} nor \textbf{ff}.

Another solution could be to allow such programs to progress to an error final configuration, displaying an adequate error message.

Our structural operational semantics is partial and deterministic (given $\sigma$ and $\delta$).
