We first identify a set of runtime errors that we would like to detect:

\begin{itemize}
	\itemsep0em
	\item[--] Because of references subtraction, one could try to access the memory with a strictly negative address.
	\item[-] A variable may be used before it is initialized, which could lead to undefined behavior.
	\item[--] If we wish to add the division operator, we may want to detect divisions by zero.
\end{itemize}

We now focus on the first error: invalid memory addresses.
We give an abstract domain for detecting such errors as well as a static analysis, which we then apply to some examples.
Regarding the other errors mentioned, we give later an overview of similar strategies that we could employ to detect them.

\subsection*{Strictly negative memory addresses}

We define the lattice $\mathcal{L} = (E, \subset, \sqcup, \sqcap)$ as illustrated in \figurename~\ref{fig:lattice} to approximate the domain range of the variable values.

\begin{figure}
	\centering
	\begin{tikzpicture}[node distance=2cm]
		\node (Z) at (0,0) {$\mathbb{Z}$};
		\node[below left of=Z] (ZP) {$\mathbb{Z}^{+}$};
		\node[below right of=Z] (ZM) {$\mathbb{Z}^{-}$};
		\node[below left of=ZP] (ZPS) {$\mathbb{Z}^{+*}$};
		\node[below right of=ZM] (ZMS) {$\mathbb{Z}^{-*}$};
		\node[below left of=ZM] (ZE) {$\lbrace 0 \rbrace$};
		\node[below of=ZE] (N) {$\bot_{int}$};
		\draw[-] (Z) -- (ZP) -- (ZPS) -- (N) -- (ZMS) -- (ZM) -- (Z) (ZE) -- (ZP) (N) -- (ZE) -- (ZM);
	\end{tikzpicture}
	\caption{Abstract domain for detecting invalid memory addresses}
	\label{fig:lattice}
\end{figure}

We now define our static analysis.
At each program point, we define a dictionary that gives an approximation domain for each variable: $\alpha: \var{} \longrightarrow E$.
This dictionary has a finite definition set, which is handy.
For this reason, we do not consider memory values that we simply abstract to the domain $\mathbb{Z}$ since the number of valid addresses is infinite.

If we have context $\Gamma$ and a program $c$ such that $\Gamma \vdash c$, then we define $\alpha^0$ with respects to $\Gamma$.

For each program point, we now define an \emph{in} and \emph{out} analysis using the control flow graph of the program $c$ as follows:

\[
\forall x \in \var{} \quad \alpha_{in}^l(x) \mapsto
	\begin{cases}
		\alpha^0(x) & \text{if $l = init(c)$} \\
		\bigsqcup\limits_{l' \rightarrow l} \alpha_{out}^{l'}(x) & \text{otherwise} \\
	\end{cases}
\]

We now define $\alpha_{out}$ in function of $\alpha_{in}$ for the basic instructions that are not captured by the control flow graph.
We use an approximation function $f: \exp{} \times (\var{} \longrightarrow E) \longrightarrow E$ that we define more precisely further below.

\begin{itemize}
	\itemsep0em
	\item[--] $\left[\texttt{skip}\right]^l$: For all $x$: $\alpha_{out}^l(x) \mapsto \alpha_{in}^l(x)$
	\item[--] $\left[x := e\right]^l$: For all $y \neq x$: $\alpha_{out}^l(y) \mapsto \alpha_{in}^l(y)$
	
	For x: $\alpha_{out}^l(x) \mapsto f(e, \alpha_{in}^l)$
	\item[--] $\left[[e_1] := e_2\right]^l$: If $f(e_1, \alpha_{in}^l)$ gives an error, we return an error.
	
	Otherwise, for all $x$: $\alpha_{out}^l(x) \mapsto \alpha_{in}^l(x)$
\end{itemize}

The function $f$ approximates the domain of the expression depending on the domains of its components and is defined as one would intuitively imagine.
A few examples are given in \tablename~\ref{tab:approx}.

\begin{table}
	\centering
	\begin{tabular}{@{}p{0.15\textwidth}p{0.3\textwidth}p{0.3\textwidth}p{0.15\textwidth}@{}}
	\toprule
	Expressions & Cases 1 & Cases 2 & Output \\
	\midrule
	$x$ & & & $\alpha_{in}^l(x)$ \\
	\midrule
	$n$ & if $\N{n} > 0$ & & $\mathbb{Z}^{+*}$ \\
	& if $\N{n} = 0$ & & $\lbrace 0 \rbrace$ \\
	\midrule
	$e_1 + e_2$ & if $f(e_1, \alpha_{in}^l) = \mathbb{Z}^{+}$ & if $f(e_2, \alpha_{in}^l) = \mathbb{Z}^{+*}$ & $\mathbb{Z}^{+*}$ \\
	& if $f(e_1, \alpha_{in}^l) = \mathbb{Z}^{+}$ & if $f(e_2, \alpha_{in}^l) = \mathbb{Z}^{-}$ & $\mathbb{Z}$ \\
	\midrule
	$e_1 - e_2$ & if $f(e_1, \alpha_{in}^l) = \mathbb{Z}^{-*}$ & if $f(e_2, \alpha_{in}^l) = \mathbb{Z}^{+}$ & $\mathbb{Z}^{-*}$ \\
	& if $f(e_1, \alpha_{in}^l) = \mathbb{Z}^{+}$ & if $f(e_2, \alpha_{in}^l) = \mathbb{Z}^{+}$ & $\mathbb{Z}$ \\
	\midrule
	$e_1 \times e_2$ & if $f(e_1, \alpha_{in}^l) = \mathbb{Z}^{-*}$ & if $f(e_2, \alpha_{in}^l) = \mathbb{Z}^{-*}$ & $\mathbb{Z}^{+*}$ \\
	& if $f(e_1, \alpha_{in}^l) = \mathbb{Z}$ & if $f(e_2, \alpha_{in}^l) = \lbrace 0 \rbrace$ & $\lbrace 0 \rbrace$ \\
	\midrule
	$[e]$ & if $f(e, \alpha_{in}^l) = \mathbb{Z}^{+}$ & & $\mathbb{Z}$ \\
	& if $f(e, \alpha_{in}^l) = \mathbb{Z}$ & & Error \\
	& if $f(e, \alpha_{in}^l) = \mathbb{Z}^{-*}$ & & Error \\
	\bottomrule
	\end{tabular}
	\caption{Approximation function $f$: a few examples}
	\label{tab:approx}
\end{table}

\paragraph{Examples}

We consider the following program: $x:= 1; y:= [x]$.
The static analysis is quite trivial and returns no error as expected.
However, if we consider the following program: $x:= 1 - 2; y:= [x]$, we have the following static analysis:

\begin{align*}
	& \alpha_{in}^1 = \lbrace x: \mathbb{Z}^{+*}, y: \mathbb{Z} \rbrace & \alpha_{out}^1 = \lbrace x: \mathbb{Z}, y: \mathbb{Z} \rbrace \\
	& \alpha_{in}^2 = \lbrace x: \mathbb{Z}, y: \mathbb{Z} & \alpha_{out}^2 \longrightarrow \text{Error} \\
\end{align*}

We obtain an error which is satisfying since the program would indeed try to use an invalid memory address.
We can however notice that the program $x:= 2 - 1; y:= [x]$ would as well return an error whereas there is actually no problem.
With this static analysis, we get some false negative and discard valid programs but we are sure all memory addresses are valid when no error is returned, \ie{} we get an over-estimation of valid programs regarding this type of errors.
 

\paragraph{Improvements} We identify two key points where our static analysis could be improved:

\begin{itemize}
	\itemsep0em
	\item[--] Memory values: all memory values are approximated by the domain $\mathbb{Z}$.
	We could be more precise by considering another function $\beta: E \longrightarrow E$ that would give a domain approximation for the value located in a range of memory addresses.
	We would need to adapt the rule for $\left[[e_1] := e_2\right]^l$ where we currently do not use $f(e_2, \alpha_{in}^l)$.
	\item[--] Boolean values: we currently consider the control flow graph but we could also give an abstract domain for booleans and define an approximation function for the different expressions.
	We then would be more precise and sometimes be able to discard branches that we are actually sure they will not be executed.
	For instance, the following program, detected currently as invalid, would pass the static analysis: $\texttt{if true then~} x:= [5] \texttt{~else~} x:= [1 - 2]$.
\end{itemize}

\subsection*{Uninitialized variables}

\todo{comments}

\subsection*{Division by zero}

\todo{comments}